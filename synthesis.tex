%%%%%%%%%%%%%%%%%%%%%%%%%%%%%
% Standard header for working papers
%
% WPHeader.tex
%
%%%%%%%%%%%%%%%%%%%%%%%%%%%%%

\documentclass[11pt]{article}

%%%%%%%%%%%%%%%%%%%%
%% Include general header where common packages are defined
%%%%%%%%%%%%%%%%%%%%



%%%%%%%%%%%%%%%%%%%%%%%%%%
%% TEMPLATES
%%%%%%%%%%%%%%%%%%%%%%%%%%


% Simple Tabular

%\begin{tabular}{ |c|c|c| } 
% \hline
% cell1 & cell2 & cell3 \\ 
% cell4 & cell5 & cell6 \\ 
% cell7 & cell8 & cell9 \\ 
% \hline
%\end{tabular}





%%%%%%%%%%%%%%%%%%%%%%%%%%
%% Packages
%%%%%%%%%%%%%%%%%%%%%%%%%%



% encoding 
\usepackage[utf8]{inputenc}
\usepackage[T1]{fontenc}


% general packages without options
\usepackage{amsmath,amssymb,amsthm,bbm}

% graphics
\usepackage{graphicx,transparent,eso-pic}

% text formatting
\usepackage[document]{ragged2e}
\usepackage{pagecolor,color}
%\usepackage{ulem}
\usepackage{soul}







%%%%%%%%%%%%%%%%%%%%%%%%%%
%% Maths environment
%%%%%%%%%%%%%%%%%%%%%%%%%%

%\newtheorem{theorem}{Theorem}[section]
%\newtheorem{lemma}[theorem]{Lemma}
%\newtheorem{proposition}[theorem]{Proposition}
%\newtheorem{corollary}[theorem]{Corollary}

%\newenvironment{proof}[1][Proof]{\begin{trivlist}
%\item[\hskip \labelsep {\bfseries #1}]}{\end{trivlist}}
%\newenvironment{definition}[1][Definition]{\begin{trivlist}
%\item[\hskip \labelsep {\bfseries #1}]}{\end{trivlist}}
%\newenvironment{example}[1][Example]{\begin{trivlist}
%\item[\hskip \labelsep {\bfseries #1}]}{\end{trivlist}}
%\newenvironment{remark}[1][Remark]{\begin{trivlist}
%\item[\hskip \labelsep {\bfseries #1}]}{\end{trivlist}}

%\newcommand{\qed}{\nobreak \ifvmode \relax \else
%      \ifdim\lastskip<1.5em \hskip-\lastskip
%      \hskip1.5em plus0em minus0.5em \fi \nobreak
%      \vrule height0.75em width0.5em depth0.25em\fi}



%%%%%%%%%%%%%%%%%%%%
%% Idem general commands
%%%%%%%%%%%%%%%%%%%%
%% Commands

\newcommand{\noun}[1]{\textsc{#1}}


%% Math

% Operators
\DeclareMathOperator{\Cov}{Cov}
\DeclareMathOperator{\Var}{Var}
\DeclareMathOperator{\E}{\mathbb{E}}
\DeclareMathOperator{\Proba}{\mathbb{P}}

\newcommand{\Covb}[2]{\ensuremath{\Cov\!\left[#1,#2\right]}}
\newcommand{\Eb}[1]{\ensuremath{\E\!\left[#1\right]}}
\newcommand{\Pb}[1]{\ensuremath{\Proba\!\left[#1\right]}}
\newcommand{\Varb}[1]{\ensuremath{\Var\!\left[#1\right]}}

% norm
\newcommand{\norm}[1]{\left\lVert #1 \right\rVert}



% argmin
\DeclareMathOperator*{\argmin}{\arg\!\min}


% amsthm environments
\newtheorem{definition}{Definition}
\newtheorem{proposition}{Proposition}
\newtheorem{assumption}{Assumption}

%% graphics

% renew graphics command for relative path providment only ?
%\renewcommand{\includegraphics[]{}}





\renewcommand{\abstractname}{}




% geometry
\usepackage[margin=2cm]{geometry}

% layout : use fancyhdr package
\usepackage{fancyhdr}
\pagestyle{fancy}

\makeatletter

\renewcommand{\headrulewidth}{0.4pt}
\renewcommand{\footrulewidth}{0.4pt}
\fancyhead[RO,RE]{\textit{Working Paper}}
\fancyhead[LO,LE]{G{\'e}ographie-Cit{\'e}s/LVMT}
\fancyfoot[RO,RE] {\thepage}
\fancyfoot[LO,LE] {\noun{J. Raimbault}}
\fancyfoot[CO,CE] {}

\makeatother


%%%%%%%%%%%%%%%%%%%%%
%% Begin doc
%%%%%%%%%%%%%%%%%%%%%

\begin{document}









\title{Vers des Modèles Couplant Développement Urbain et Croissance des Réseaux de Transport\bigskip\\
\textit{Synthèse de mi-thèse}
}
\author{\noun{Juste Raimbault}}
\date{Octobre 2016}


\maketitle

\justify


\begin{abstract}
\end{abstract}






\paragraph{Du positionnement général}

\emph{L'ambition de cette thèse est de ne pas avoir d'ambition.} Cette entrée en matière, rude en apparence, contient à différents niveaux les logiques sous-jacentes à notre processus de recherche. Au sens propre, nous nous plaçons tant que possible dans une démarche constructive et exploratoire, autant sur les plans théoriques et méthodologiques que thématique, mais encore proto-méthodologique (outils appliquant la méthode) : si des ambitions unidimensionnelles ou intégrées devaient émerger, elles seraient conditionnées par l'arbitraire choix d'un échantillon temporel parmi la continuité de la dynamique qui structure tout projet de recherche. Au sens structurel, l'auto-référence qui soulève une contradiction apparente met en exergue l'aspect central de la réflexivité dans notre démarche constructive, autant au sens de la récursivité des appareils théoriques, de celui de l'application des outils et méthodes développés au travail lui-même ou que de celui de la co-construction des différentes approches et des différents axes thématiques. Le processus de production de connaissance pourra ainsi être lu comme une métaphore des processus étudiés. Enfin, sur un plan plus enclin à l'interprétation, cela suggérera la volonté d'une position délicate liant un positionnement politique dont la nécessité est intrinsèque aux sciences humaines (par exemple ici contre l'application technocratique des modèles, ou pour le développement d'outils luttant pour une science ouverte) à une rigueur d'objectivité plus propre aux autres champs abordés, position forçant à une prudence accrue.


\paragraph{Des objectifs scientifiques}

% ambition ≠ objectif

\emph{L'objectif d'une variété.} Un objectif est différent d'une ambition, et ceux-ci sont ainsi pour nous clairement fixés sur différents aspects et à différents niveaux. L'objectif principal du point de vue du géographe est d'enrichir l'état de la connaissance sur les processus co-évolutifs entre territoires et réseaux (la définition de ces termes et l'appareil théorique associé faisant parties intégrantes des sous-objectifs), par l'entrée particulière des réseaux de transports et dans une perspective axée premièrement sur la modélisation. Les aspects géographiques peuvent se décliner en sous-objectifs sur des plans variés :
\begin{enumerate}
\item Etablir par une étude d'épistémologie qualitative et quantitative le paysage scientifique associé à notre objectif principal, notamment sa diversité lié aux disciplines variées y étant associées.
\item Extraire des faits stylisés empiriques sur les processus liant territoires et réseaux, à différentes échelles temporelles et spatiales et sur différents cas d'étude.
\item Construire des modèles de croissance urbaine et/ou de croissance des réseaux, pouvant aller du modèle jouet au modèle semi-paramétrisé, dans le but d'être soit des outils exploratoires soit des briques élémentaires d'une famille de modèles de co-évolution des réseaux et des territoires.
\item Par émergence issue de l'interaction des objectifs précédents, élaborer une théorie géographique des \emph{systèmes territoriaux réticulaires co-évolutifs}.
\end{enumerate} 

Des objectifs dont les aspects pouvant être classifiés à dominante plutôt méthodologique ou proto-méthodologique (même s'il est clair que dans la pratique l'ensemble des objectifs est complémentaire et entrelacé de manière \emph{complexe}) viennent ensuite s'ajouter : 

\begin{enumerate}\setcounter{enumi}{4}
\item Exercices de style sur différentes questions horizontales fondamentales à l'étude des système complexes, liés de près ou de loin au sujet thématique, dans le but d'un apport méthodologique.
\item Développement d'outils (libres et ouverts) et de techniques, que ce soit au niveau de problèmes précis ou au niveau de l'organisation générale du travail de recherche.
\end{enumerate}

Enfin, 







\paragraph{Du contenu courant}

\emph{L'auto-organisation prend souvent l'architecture de court.} Une grande partie du travail résumé ci-dessous est organisé sous forme provisoire dans \cite{raimbault2016memoire} qui peut être lu comme complément à cette synthèse


\paragraph{Du contenu final}

\emph{La route est longue mais la voie est libre.}







%%%%%%%%%%%%%%%%%%%%
%% Biblio
%%%%%%%%%%%%%%%%%%%%

\bibliographystyle{apalike}
\bibliography{/Users/Juste/Documents/ComplexSystems/CityNetwork/Biblio/Bibtex/CityNetwork}


\end{document}
